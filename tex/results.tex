%!TEX root = ../paper.tex

The definition of Fisher's information metric is
\begin{equation}\label{eq:fisherinformationmetric}
    g_{\theta \theta} = \sum_{i} p_i \frac{\partial \log p_i}{\partial \theta} \frac{\partial \log p_i}{\partial \theta}.
\end{equation}
The binomial distribution is
\begin{equation}\label{eq:binomialdist}
    \begin{aligned}
        p(n) &= \binom{N}{n} \theta^n (1-\theta)^{N-n},\\
        &\approx \frac{1}{\sqrt{2\pi N \theta(1-\theta)}} e^{-\frac{(n-\theta N)^2}{2 N \theta(1-\theta)}}
    \end{aligned}
\end{equation}
For the binomial distribution of \ref{eq:binomialdist}, we have parameters $p_1 = \theta,\,p_2 = 1 - \theta$, and \ref{eq:fisherinformationmetric} takes the form
\begin{equation}\label{eq:fimbinomial}
    \begin{aligned}
        g_{\theta \theta} &= \theta \left( \frac{\partial}{\partial \theta} \log \theta \right)^2 + (1 - \theta) \left( \frac{\partial}{\partial \theta} \log (1 - \theta) \right)^2,\\
                          &= \frac{1}{\theta} + \frac{1}{1 - \theta},\\
                          &= \frac{1}{\theta (1-\theta)}.
     \end{aligned}
 \end{equation}
The arc length is then given in this case as
\begin{equation}\label{eq:arclengthbinomial}
    s = \int d \theta \sqrt{g_{\theta \theta}} = \int \frac{d\theta}{\sqrt{\theta(1-\theta)}}
\end{equation}
Introducing the frequencies $f_i = \frac{n_i}{N}$ we can write the standard deviation as
\begin{equation}
    \begin{aligned}
        \sqrt{\langle f^2 \rangle - \langle f \rangle^2} &= \sqrt{\frac{1}{N^2} \left(\langle n^2 \rangle - \langle n \rangle^2 \right)}\\
        &= \sqrt{\frac{\theta(1-\theta)}{N}}
    \end{aligned}
\end{equation}
In general, an exponential family is one which can be parametrized such that
\begin{equation}\label{eq:expfamily}
    p(x,\theta) = e^{\sum_{i=1}^{r} \theta^i x^i - K(\theta)}
\end{equation}
For the case of a single binomial random variable $x \in \{0,1\}$ this should result in the form
\begin{equation}
    p(\theta)(x) = e^{\theta x - K(\theta)}.
\end{equation}
Substituting in possible values of the random variable and normalizing indicates that $K(\theta) = \log (1+e^{\theta})$ and thus we have
\begin{equation}\label{eq:binexpfampars}
    \begin{aligned}
        p_0 &= \frac{1}{1+e^{\theta}},\\
        p_1 &= \frac{e^{\theta}}{1+e^{\theta}}.
    \end{aligned}
\end{equation}
Substituting these into \ref{eq:fisherinformationmetric} yields
\begin{equation}\label{eq:fimbinomialexppars}
    \begin{aligned}
        g_{\theta \theta} &= \frac{1}{1+e^{\theta}} \left( \frac{\partial}{\partial \theta} \log \frac{1}{1+e^{\theta}} \right)^2 + \frac{e^{\theta}}{1+e^{\theta}} \left( \frac{\partial}{\partial \theta} \log \frac{e^{\theta}}{1+e^{\theta}} \right)^2,\\
                          &= \frac{e^{\theta}}{(1+e^{\theta})^2}.
     \end{aligned}
 \end{equation}
In these terms the analog of the arc length in \ref{eq:arclengthbinomial} is
\begin{equation}\label{eq:arclengthbinomialexppar}
    s = \int d \theta \sqrt{g_{\theta \theta}} = \int \frac{e^{\frac{1}{2} \theta} d\theta}{1+e^{\theta}}
\end{equation}
The metric \ref{eq:fimbinomialexppars} is invariant to fractional linear transformations of the form
\begin{equation}
    (\mu_i) \mapsto \left( \frac{e^{f_i}\mu_i}{\sum_i e^{f_i}\mu_i} \right).
\end{equation}
If we take $(\mu_0,\mu_1)$ equal to $(p_0,p_1)$ from \ref{eq:binexpfampars} this transformation results in
\begin{equation}
    \left(\frac{1}{1+e^{\theta}}, \frac{e^{\theta}}{1+e^{\theta}} \right)
    \mapsto
    \left(\frac{e^{f_0}}{e^{f_0}+e^{f_1+\theta}}, \frac{e^{\theta+f_1}}{e^{f_0}+e^{\theta+f_1}} \right)=\left(\frac{1}{1+e^{\theta - f_0 + f_1}}, \frac{e^{\theta - f_0 + f_1}}{1+e^{\theta - f_0 + f_1}} \right),
\end{equation}
which corresponds to a transformation of parameters $\theta \mapsto \theta-f_0+f_1$. This clearly corresponds to the identity when $f_0=f_1$.
